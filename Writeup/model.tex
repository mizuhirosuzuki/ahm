\documentclass[12pt, fleqn]{article}

\usepackage{amsmath}
\usepackage{amsthm}
\usepackage{amssymb}
\usepackage{graphicx}
\usepackage{longtable}
\usepackage{booktabs}
\usepackage{fullpage}
\usepackage{lscape}
\usepackage{natbib}
\usepackage{setspace} 
\usepackage{mathtools}
\doublespacing

\begin{document}
\title{Agricultural Household Model: Structural Estimation (Monte Carlo Simulation)}
\author{Mizuhiro Suzuki}
\maketitle

\section{Basic Model}

I consider the model in which a farmer maximizes lifetime expected utility by making decisions in agricultural production and household.
More specifically, the farmer solves the following maximization problem by deciding (i) consumption, (ii) production input, and (iii) saving:
\begin{align*}
  &\max_{\{C_t, M_t, S_t\}_{t=0}^{\infty}} U(W_t) = \sum_{t=0}^{\infty} E_0\left[\frac{C_t^{1-\rho}}{1-\rho}\right] \\
        &\text{subject to} \\
        & \quad C_t + M_t + S_t = W_t                  \quad \text{(budget constraint)} \\
        & \quad Y_{t+1} = M_t^{\beta_m} \varepsilon_{t+1} \quad \text{(production function)} \\
        & \quad W_{t+1} = Y_{t+1} + Z_{t+1} + (1+r) S_t   \quad \text{(wealth transition)},
\end{align*}

where
\begin{itemize}
  \item $C_t$: Consumption, $M_t$: Production inputs, $W_t$: Wealth
  \item $Y_t$: Business revenue, $Z_t \sim \log N(\mu_Z, \sigma_Z)$: ``Other income''
  \item $\varepsilon_t \sim \log N(\mu_{\varepsilon}, \sigma_{\varepsilon})$: Productivity shocks
\end{itemize}

The Bellman equation is
\begin{align*}
  V(W) &= \max_{C, M, S} \frac{C^{1-\rho}}{1-\rho} + \beta E\left[V(W') \right] \\
        &\text{subject to} \\
        & \quad C + M + S = W                 \quad \text{(budget constraint)} \\
        & \quad Y' = M^{\beta_m} \varepsilon' \quad \text{(production function)} \\
        & \quad W' = Y' + Z' + (1+r) S        \quad \text{(wealth transition)},
\end{align*}

The first order conditions are
\begin{align*}
  C^{-\rho} &= \beta \beta^m M^{\beta_m-1} E[\varepsilon' V'(M^{\beta_m} \varepsilon' + Z' + (1+r) S)] \\
  C^{-\rho} &= \beta (1+r) M^{\beta_m-1} E[V'(M^{\beta_m} \varepsilon' + Z' + (1+r) S)]
\end{align*}
and the envelope condition is
\begin{equation*}
  V'(W) = (W - S - M)^{- \rho}.
\end{equation*}

Note that there is an uncertainty in production but there is not insurance for that risk.
Hence, separability does not hold and the household preference (in this case, risk preference) affects production decision making (in this case, production input).

\section{Estimation}
Parameters to be estimated are $\theta = [\rho, \beta, \beta_m, \mu_{\varepsilon}, \sigma_{\varepsilon}\, \mu_{Z}, \sigma_{Z}]$.
I use the indirect inference to estimate these parameters.
Estimation involves the following steps:
\begin{enumerate}
  \item Given a set of parameters ($\theta$), solve the dynamic programming model
  \item Simulate the data and obtain simulated moments ($m(\theta)$)
  \item Obtain the distance between simulated moments and data moments ($m$)
  \item Repeat 1-3 to minimize the distance: $\widehat{\theta} = \text{argmin}_{\theta} \ (m(\theta) - m)' (m(\theta) - m)$
\end{enumerate}

Appendix A shows the moment conditions used for the identification.
Roughly speaking, the parameters and moment conditions correspond to each other in the following way:
\begin{align*}
  \rho, \beta                                               & \leftrightarrow \text{Consumption} \\
  \beta_m, \mu_{\varepsilon}, \sigma_{\varepsilon}          & \leftrightarrow \text{Production and sales} \\
  \mu_{Z}, \sigma_{Z}                                       & \leftrightarrow \text{Expenditure and sales}
\end{align*}

\clearpage

%\bibliographystyle{apalike}
%\bibliography{se}

\appendix
\section{Moment conditions}
\begin{align*}
    E[E[C|K, W] - C]                                      =  0 \\
    E[E[(C - E[C|K, W])^2 | K, W] -  (C - \overline{C})^2]     =  0 \\
    E[(E[C|K, W] - C)W]                                   =  0 \\
    E[(E[C|K, W] - C)K]                                   =  0 \\
    E[E[M|K, W] - M]                                      =  0 \\
    E[E[(M - E[M|K, W])^2 | K, W] -  (M - \overline{M})^2]     =  0 \\
    E[(E[M|K, W] - M)W]                                   =  0 \\
    E[(E[M|K, W] - M)K]                                   =  0 \\
    E[E[D|K, W] - D]                                      =  0 \\
    E[E[Y'|K, W] - Y']                                    =  0 \\
    E[E[(Y' - E[Y'|K, W])^2 | K, W] -  (Y' - \overline{Y'})^2] =  0 \\
    E[E[W'|K, W] - W']                                    =  0 \\
    E[E[(W' - E[W'|K, W])^2 | K, W] -  (W' - \overline{W'})^2] =  0 \\
    E[E[K'|K, W] - K']                                    =  0 \\
    E[E[(K' - E[K'|K, W])^2 | K, W] -  (K' - \overline{K'})^2] =  0 
\end{align*}



\end{document}



